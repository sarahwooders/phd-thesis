\section{Conclusion}
\label{s:conclusions}

In this paper, we studied the challenge of feature maintenance in feature stores, an emerging new class of systems.
First, we identified a critical limitation in existing approaches to feature store design: current feature stores treat data and keys symmetrically and do not leverage crucial signal about query access patterns or the impact of features on downstream task performance.
We then formalized the feature store problem and introduced \textit{feature store regret}, a metric that measures the impact of staleness of features on the downstream prediction accuracy.
Finally, we presented \system{}, a feature store system that uses prediction loss as feedback to prioritize updates that improve the downstream accuracy via Regret-Proportional scheduling.
Experiments on a range of feature maintenance policies demonstrate that prioritizing replacing features with the highest \textit{cumulative regret} can significantly improve prediction accuracy in resource-constrained settings. We believe this paper will provide a formal foundation for a key problem in the emerging class of feature store systems and hope that it will inspire future work in the design of more advanced feature maintenance strategies.
% \begin{itemize}
%     \item Co-scheduling queries and updates 
%     \item Sharing state 
% \end{itemize}
