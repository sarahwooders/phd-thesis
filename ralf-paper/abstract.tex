\begin{abstract}
Feature stores (also sometimes referred to as embedding stores) are becoming ubiquitous in model serving systems: downstream applications query these stores for auxiliary inputs at inference-time. Stored features are derived by \textit{featurizing} rapidly changing base data sources. Featurization can be costly prohibitively expensive to trigger on every data update, particularly for features that are vector embeddings computed by a model. Yet, existing systems naively apply a one-size-fits-all policy as to when/how to update these features, and do not consider query access patterns or impacts on prediction accuracy. 
This paper introduces \system{}, which orchestrates feature updates by leveraging \textit{downstream error feedback} to minimize \textit{feature store regret}, a metric for how much featurization degrades downstream accuracy. We evaluate with representative feature store workloads, anomaly detection and recommendation, using real-world datasets. We run system experiments with a 275,077 key anomaly detection workload on 800 cores to show up to a 32.7\% reduction in prediction error or up to $1.6\times$ compute cost reduction with accuracy-aware scheduling. 
%\textcolor{red}{For resource constrained settings, we demonstrate up to 30\% improvements in downstream prediction accuracy for an information retrieval task, and loss reductions of up to 23\% for estimating time-series anomalies.}
\end{abstract}
